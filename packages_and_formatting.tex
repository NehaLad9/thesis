% clean start

% Choose the language
\ifxetexorluatex
	\usepackage{polyglossia}
	\setmainlanguage{english}
\else
	\usepackage[ngerman, english]{babel} % Load characters and hyphenation
\fi
\usepackage[english=american]{csquotes}	% English quotes

\usepackage{enumitem}


%use to have line numbers 
% \usepackage{lineno}
% \linenumbers
\usepackage{makecell}
\usepackage{hyperref}
\usepackage{longtable}
\usepackage{booktabs}
\usepackage{tabularx}
\usepackage{siunitx}
\sisetup{
   separate-uncertainty
}

\usepackage{physics}
\usepackage{courier}
\usepackage{amstext} % for \text macro
\usepackage{array}   % for \newcolumntype macro
\newcolumntype{L}{>{$}l<{$}} % math-mode version of "l" column type

% Load the bibliography package
\usepackage[giveninits=true]{kaobiblio}
\addbibresource{references.bib} % Bibliography file

\ifdefined\rm
\renewcommand{\rm}{\mathrm}
\else
\newcommand{\rm}{\mathrm}
\fi

\ifdefined\gev
\renewcommand{\gev}{\giga\electronvolt}
\else
\newcommand{\gev}{\giga\electronvolt}
\fi

\ifdefined\mev
\renewcommand{\mev}{\mega\electronvolt}
\else
\newcommand{\mev}{\mega\electronvolt}
\fi

% define HNL mixing abbreviations
\def\ue4{$|U_{e4}|^2$}
\def\um4{$|U_{\mu4}|^2$}
\def\ut4{$|U_{\tau4}|^2$}



% Load mathematical packages for theorems and related environments
\usepackage[framed=true]{kaotheorems}
\usepackage{amsmath}

% Load the package for hyperreferences
\usepackage{kaorefs}

\usepackage{caption}
\usepackage{subcaption}

\graphicspath{{./}{figures/}} % Paths in which to look for images

% \makeindex[columns=3, title=Alphabetical Index, intoc] % Make LaTeX produce the files required to compile the index


% for Alex' crazy tikz stuff...
\usepackage{pgfplots}
\usepackage{pgfplotstable}
\usepgfplotslibrary{fillbetween}

\ifxetexorluatex
    % do nothing
\else
    \DeclareUnicodeCharacter{2212}{−}
\fi

\usepgfplotslibrary{groupplots,dateplot}
\usetikzlibrary{patterns,shapes,arrows}
\usetikzlibrary{fadings}
\pgfplotsset{compat=newest}

\usepackage{feynmp-auto}
\usepackage[compat=1.1.0]{tikz-feynman}

% global color definitions
\definecolor{black}{RGB}{0,0,0}
\definecolor{orange}{RGB}{230, 159, 0}
\definecolor{skyblue}{RGB}{86, 180, 233}
\definecolor{bluishgreen}{RGB}{0, 158, 115}
\definecolor{yellow}{RGB}{240, 228, 66}
\definecolor{blue}{RGB}{0, 114, 178}
\definecolor{vermilion}{RGB}{213, 94, 0}
\definecolor{reddishpurple}{RGB}{204, 121, 167}

\definecolor{tab_orange}{RGB}{255, 128, 14}
\definecolor{tab_blue}{RGB}{0, 107, 164}

\definecolor{lightgray204}{RGB}{204,204,204}

\tikzstyle{nue_color}=[orange]
\tikzstyle{numu_color}=[bluishgreen]
\tikzstyle{nutau_color}=[skyblue]
% color for any "all NC" plots
\tikzstyle{nc_color}=[reddishpurple]
\tikzstyle{noise_color}=[blue]
\tikzstyle{muon_color}=[vermilion]
% color for all combined nu flavors
\tikzstyle{nu_color}=[bluishgreen]

% plot an error band assuming that a table has been loaded.
% Syntax: \ploterrorband[style]{column_name}{scale_factor}
% The column name and the column "column_name__err" must exist. The
% scale factor scales the entire error band at once.
\newcommand{\ploterrorband}[3][black]{
	\addplot[const plot, #1, thick] table [x=bin_edges, y expr=\thisrow{#2} * #3] from \table;
    \addplot[const plot, #1, thin, name path=err_lo, forget plot] table[x=bin_edges, y expr=(\thisrow{#2}-\thisrow{#2__err}) * #3] from \table;
    \addplot[const plot, #1, thin, name path=err_hi, forget plot] table[x=bin_edges, y expr=(\thisrow{#2}+\thisrow{#2__err}) * #3] from \table;
    \addplot[#1, opacity=0.5, forget plot] fill between[of = err_lo and err_hi];
}

\newcommand{\plotratioerrorband}[3][black]{
	\addplot[const plot, #1, thick] table [x=bin_edges, y expr=\thisrow{#2} / \thisrow{#3}] from \table;

    \addplot[const plot, #1, thin, name path=err_lo, forget plot] table[x=bin_edges,
    y expr=
    	\thisrow{#2} / \thisrow{#3}
		-
		sqrt(
            (\thisrow{#2} * \thisrow{#3__err})^2 + (\thisrow{#3} * \thisrow{#2__err})^2 - (2 * \thisrow{#2} * \thisrow{#3} * \thisrow{#2__err}^2)
		) / \thisrow{#3}^2
	] from \table;

	\addplot[const plot, #1, thin, name path=err_hi, forget plot] table[x=bin_edges,
    y expr=
    	\thisrow{#2} / \thisrow{#3}
		+
		sqrt(
            (\thisrow{#2} * \thisrow{#3__err})^2 + (\thisrow{#3} * \thisrow{#2__err})^2 - (2 * \thisrow{#2} * \thisrow{#3} * \thisrow{#2__err}^2)
		) / \thisrow{#3}^2
	] from \table;

    \addplot[#1, opacity=0.5, forget plot] fill between[of = err_lo and err_hi];
}


\pgfplotsset{error bar legend/.style={%
    /pgfplots/legend image code/.prefix code={%
      \pgfkeysgetvalue{/pgfplots/error bars/error mark}{\pgfplotserrorbarsmark}%
      \draw[%
        /pgfplots/every error bar,
        mark=\pgfplotserrorbarsmark,
        /pgfplots/error bars/error mark options,
        sharp plot,
        ##1
      ] plot coordinates {(0.3cm, -0.1cm) (0.3cm, 0.1cm)};%
    }
  }
}


\tikzstyle{block} = [rectangle, rounded corners, minimum width=3cm, minimum height=1cm, text centered, draw=black, fill=gray!20]
\tikzstyle{decision} = [ellipse, aspect=2, minimum width=3cm, minimum height=1cm, text centered, draw=black, fill=gray!20]
\tikzstyle{arrow} = [thick,->,>=stealth]

\newcommand{\ploterrorbar}[2][black]{
    \addplot[
        mark=*,
        mark options={scale=0.5, fill=black},
        #1,
        only marks,
        error bar legend,
        error bars/.cd,
        x dir=none,
        y dir=both,
        y explicit
    ] table [x=bin_midpoints, y=#2, y error=#2__err]  from \table;
}

\newcommand\barp[1]{\overset{%
\scriptscriptstyle(-)}{#1}}

% This package changes the font to serif when compiling with Lualatex (and doesn't)
% work correctly without Lualatex.
%\usepackage{helvet}
\usepackage[eulergreek]{sansmath}
\pgfplotsset{
    tick label style = {font=\footnotesize\sansmath\sffamily},
    every axis label = {font=\footnotesize\sansmath\sffamily},
    % the eulergreek style for math looks oddly different from the
    % rest of the document, so we omit \sansmath here
    label style = {font=\footnotesize\sffamily},
    % global legend style applied to all plots (set per plot to override)
    legend style = {
        font=\footnotesize\sffamily,
        fill opacity=0.8,
        draw opacity=1,
        text opacity=1,
        at={(0.5,0.95)},
        anchor=north,
        draw=lightgray204,
        % hack to get better legend spacing, see
        % https://tex.stackexchange.com/questions/18152/how-can-i-adjust-the-horizontal-spacing-between-legend-entries-in-pgfplots
        /tikz/every even column/.append style={column sep=0.1cm}
    },
    legend cell align={left}
}

\tikzstyle{plot annotation}=[
    font=\footnotesize\sffamily,
    fill opacity=0.8,
    draw opacity=1,
    text opacity=1,
    draw=lightgray204,
    fill=white
]

\usepackage{tikz}
\newcommand\wye[1][]{%
    \tikz\draw[thick, line cap=round,x=1ex,y=1ex,#1]
    (0,0) -- ++(90:1)
    (0,0) -- ++(-30:1)
    (0,0) -- ++(-150:1);
}