\chapter{Zusammenfassung}
Kosmische Strahlung und hochenergetische Neutrinos haben eine grundlegende Verbindung, da beide in extremen astrophysikalischen Umgebungen entstehen, in denen kosmische Strahlung mit Materie oder Photonen in der Beschleunigungsregion interagiert und Neutrinos erzeugt. Die Beobachtung der Flavor-Zusammensetzung von Neutrinos bietet wertvolle Einblicke in diese Produktionsmechanismen. Aufgrund der Neutrinooszillationen über kosmische Entfernungen werden Tau-Neutrinos - die bei der Produktion nicht vorhanden sind - erwartet, wenn die Neutrinos die Erde erreichen, sodass ihr Nachweis für die Untersuchung astrophysikalischer Neutrinoquellen und der Flavor-Entwicklung entscheidend ist.

In dieser Arbeit wird die Flavor-Zusammensetzung hochenergetischer astrophysikalischer Neutrinos gemessen, wobei der Schwerpunkt auf der Entdeckung von Tau-Neutrinos in einer Stichprobe hochenergetischer Startereignisse liegt, die über zwölf Jahre hinweg vom IceCube-Neutrino-Observatorium gesammelt wurden. Mithilfe eines Klassifizierers werden die hochenergetischen Startereignisse in drei Kategorien eingeteilt - Einzelkaskade, Doppelkaskade, und Spur -, die in erster Linie mit den Neutrino-Flavors (\( \nu_e \), \( \nu_\tau \) und \( \nu_\mu \)) in Verbindung stehen. Fünf Ereignisse wurden als Doppelkaskaden identifiziert, darunter vier bisher unbeobachtete, neben 64 Einzelkaskaden und 28 Startspuren, die mit den Erwartungen an ein einfaches Potenzgesetz-Neutrinospektrum übereinstimmen. Diese Arbeit beinhaltet signifikante Simulations- und Rekonstruktionsverbesserungen, einschließlich Anpassungen an Wirkungsquerschnitte, Tau-Lepton-Polarisation, aktualisierte Neutrinogewichtung und die neuesten Südpol-Eis-Modelle, um die anisotrope Lichtausbreitung besser zu erfassen und so die Reinheit der Doppelkaskadenauswahl zu erhöhen.

Die auf diesen Kategorien basierende Flavourzusammensetzung sind \( f_{\nu_e} : f_{\nu_{\mu}} : f_{\nu_{\tau}} = 0.19_{-0.15}^{+0.26} : 0.43_{-0.17}^{+0.27} : 0.38_{-0.24}^{+0.37} \) mit 68\% Unsicherheiten. Die Ergebnisse sprechen leicht für ein Szenario mit muongedämpftem Flavor (0:1:0) an den Quellen, obwohl alternative Szenarien innerhalb eines 1-Sigma-Bereichs möglich sind. Obwohl keine signifikanten Ausschlüsse von Quellenmodellen erreicht wurden, zeigen die Verbesserungen bei der Rekonstruktion und Simulation wichtige Einschränkungen auf, insbesondere bei der Identifizierung von Doppelkaskaden. Diese Arbeit schafft eine Grundlage für die Verfeinerung der Rekonstruktionen von Doppelkaskaden bei zukünftige IceCube-Analysen.

Darüber hinaus wurden Empfindlichkeitsprojektionen für IceCube-Gen2, die geplante Erweiterung des IceCube-Detektors, erstellt, um den Nachweis von hochenergetischen Tau-Neutrinos zu verbessern und die Flavor-Zusammensetzung mit Multi-PMT-Module und über ein erweitertes Detektorvolumen zu messen, die als isotrope sphärische Sensoren modelliert wurden. Die Ergebnisse deuten auf ein großes Potenzial hin, verschiedene Quellenszenarien auszuschließen und eine energieabhängige Flavor-Zusammensetzung zu bestimmen, was Studien zur Quellenentwicklung über das gesamte diffuse Neutrinospektrum hinweg unterstützt.