\chapter{Introduction}
\labch{intro}
The study of high-energy cosmic phenomena has long been driven by the quest to understand the processes that produce extreme energies in the universe. High-energy astrophysical events provide insights into the
most energetic processes, from the cores of active galaxies to supernova explosions. Early observations of cosmic rays marked humanity's first detection of high-energy particles from beyond Earth, but interpreting
these signals remained a challenge due to the charged nature of cosmic rays, which makes their path susceptible to magnetic fields, masking their origins. This challenge initiated the search for alternative messengers (that were not just photons), giving rise to the field of \textbf{astroparticle physics}—a field that blends astronomy, particle physics, and astrophysics to probe the high-energy

The history of our exploration into high-energy astrophysics is marked by a series of pivotal moments and discoveries. In the early 20th century, the first hints of cosmic rays were detected by Victor Hess, who observed the elevation of radiation levels at high altitudes, indicating an extraterrestrial origin for cosmic rays \sidecite{HESS_Balloon}. As the field matured, the concept of astroparticle physics emerged, characterized by its multidisciplinary approach that blends astrophysics, particle physics, and cosmology. The pioneering experiment conducted by Ray Davis and his team at the Homestake Mine successfully detected neutrinos produced by nuclear fusion in the solar core, an achievement that not only confirmed theories of solar energy production but also initiated a new era in neutrino astronomy \sidecite{homestake}. This was further exemplified by the detection of neutrinos from the supernova SN1987A, which provided a direct observation of a stellar explosion and highlighted the importance of neutrinos as messengers of cosmic events \sidecite{SN1987A_superK,SN1987A_Baksan,SN1987A_IMB}. 

The historical narrative of neutrinos is fascinating, tracing back to their theoretical postulation by Wolfgang Pauli in 1930, who sought to resolve the apparent violation of conservation laws in beta decay \sidecite{Pauli}. Following their prediction, neutrinos remained elusive until their experimental detection in the 1950s by Clyde Cowan and Frederick Reines \sidecite{nu_discovery}. Over the subsequent decades, significant strides were made in understanding neutrino properties, including the revelation of their oscillation phenomena \sidecite{Pontecorvo}, which indicated that neutrinos have mass—a groundbreaking finding that necessitated a revision of the Standard Model of particle physics. The detection of solar neutrinos not only validated the models of nuclear fusion in stars \sidecite{Bahcall} but also introduced the concept of flavor oscillations, shedding light on the relationships between different neutrino types and their behavior as they propagate through space \sidecite{Ahmad_2001}.

The emergence of multi-messenger astronomy has transformed our understanding of the universe, allowing us to draw connections between cosmic rays, gamma rays, neutrinos, and gravitational waves. This approach is particularly significant in light of monumental observations such as the detection of gravitational waves from binary black hole mergers by LIGO \sidecite[-1cm]{LIGO_mergerpaper} and the identification of high-energy neutrinos from a blazer by the IceCube Neutrino Observatory \sidecite{txspaper}. Each messenger complements the others, providing a more nuanced picture of high-energy phenomena. For instance, the detection of neutrinos from astrophysical sources, such as blazars and starbust galaxy, has underscored the potential for neutrino astronomy to elucidate the processes that generate cosmic rays and gamma rays. In this context, the contributions of neutrino astronomy are invaluable, as they allow us to probe the inner workings of some of the most energetic events in the universe.

The evolution of our understanding of neutrinos parallels significant advancements in detector technology, reflecting the challenges and triumphs of the field. Neutrinos are unique messengers due to their weak interactions with matter, making them exceedingly difficult to detect. However, this very property also makes them powerful tools for studying fundamental physics, as they can escape dense astrophysical environments that would otherwise obscure other forms of radiation. Over the years, innovative detection methods have emerged, such as the development of large-scale neutrino telescopes that utilize the Cherenkov radiation produced when neutrinos interact with matter. The conceptual introduction of neutrino telescopes began with early experiments like DUMAND \sidecite[-7cm]{PhysRevD.42.3613} and AMANDA \sidecite[-6cm]{ANDRES20001}, leading to the realization of the IceCube Neutrino Observatory at the South Pole \sidecite[-5cm]{Halzen:2010yj}. As a result of these advancements, the capabilities for neutrino detection have increased dramatically, allowing for the exploration of a broader range of astrophysical phenomena.

\textbf{The IceCube Neutrino Observatory} has achieved remarkable milestones since it started to operate fully since 2011, establishing itself as one of the most remarkable neutrino telescopes in the field of neutrino astronomy. Among its significant accomplishments is the discovery of astrophysical neutrinos, which marked a turning point in the field, confirming that high-energy neutrinos originate from cosmic sources rather than being solely produced in Earth's atmosphere \sidecite[-6cm]{Evidence_paper}. The observation of a correlation between high-energy neutrinos from the TXS 0506+056 blazar and its gamma-ray emission demonstrated the connection between different messenger particles \cite{txspaper}. Additionally, IceCube has made important contributions to the study of neutrinos from various sources, such as the sayfert galaxy NGC 1068 \sidecite[-7cm]{ngc1068} and also neutrinos from our own galaxy, Milky Way \sidecite[-6cm]{icecube_milkyway}. In addition to these sources, and discovery of the diffuse neutrino flux, IceCube has also been able to measure the features of the neutrino flux with striking accuracies via different event samples, that target various neutirno flavours \sidecite[-6cm]{cscd_6yr,diffusenumu,HESE7_sample, ESTES}. The most recent results, combining some of these data samples and another independent sample revealed a spectral break in the diffuse neutrino spectrum with >$4\sigma$ confidence \sidecite[-1.5cm]{globalfit_icrc, MESE_ICRC}. Such features in the spectrum, can help recognising what are the dominant sources at different energy ranges. Furthermore, IceCube has placed tight constraints on neutrino oscillation parameters by measuring atmospheric tau neutrino appearance \sidecite{IceCube_atm_numixing,IceCube:2024xjj} contributing to our understanding of fundamental particle interactions.  

In addition to IceCube, there also exists a water-based neutrino telescope in Mediterranean sea at present \textbf{the Astronomy With A Neutrino Telescope And Abyss Environmental Research experiment (ANTARES)} \sidecite{AGERON201111} (now decomissioned) and its sucessor (under construction) \textbf{A kilometercube Neutrino telescope (KM3NeT)} \sidecite{MARGIOTTA201483}. The IceCube detector has also proposed for its successor, called \textbf{The IceCube Gen2}, which will be about 8 times bigger than its current volume and will also host a radio detector array, to measure very high energy cosmic neutrinos via \emph{radio detection methods} \sidecite{Gen2_TDR}. These radio techniques to detect high energy neutrino has been used in experiments like \textbf{The Radio Neutrino Observatory Greenland (RNO-G)} \sidecite{rnog}, which benefitted greatly from other radio neutrino experments such as \textbf{the Askaryan Radio Array (ARA)} \sidecite{ARA} and \textbf{the Antarctic Ross Ice Shelf Antenna Neutrino Array (ARIANNA)} \sidecite{ARIANNA} experiments. These experiments utilize the Askaryan effect as detection principle, \sidecite{Askaryan}, that will significantly expand the energy range of neutrino observatories, unlocking new insights into the world of high energy astroparticle physics.

While the achievements of IceCube are commendable, the study of tau neutrinos—one of the three flavors of neutrinos—has garnered particular attention due to their limited appearances at high energies. Tau neutrinos were postulated as a theoretical necessity with the discovery of the tau lepton in 1975 \sidecite{PhysRevLett.35.1489,PERL1977487}, yet they remained undetected for decades, until 2001 by DONUT collaboration \sidecite{DONUT}. Their only recent observations raise significant questions about the sources of neutrinos. Unlike their electron and muon counterparts, tau neutrinos do not get produced in high energy sources or in atmospheric interactions of cosmic rays or terrestrial sources\footnote{aside from those produced from charmed mesons in cosmic ray interactions.}. Although neutrino oscillations guarantees a non-zero tau neutrino fluxes. Depending on matter and magnetic field environments at the particle accelerator sites, the neutrino production may happen with different fractions \sidecite{cite170,cite168}, meaning the most straightforward scenario where due to decays of pions and subsequent decays of produced muons and electrons can produce neutrinos with flavour ratios 1:2:0 (or $\frac{1}{3}:\frac{2}{3}:0$), which due to oscillations gets converted into $\frac{1}{3}:\frac{1}{3}:\frac{1}{3}$ on earth\footnote{assuming three flavour of neutrinos}. The ability to detect tau neutrinos is hence critical for measuring both the flavor composition of astrophysical neutrinos at high energies and the oscillation parameters at lower energies. Understanding tau neutrinos opens new avenues in exploring the fundamental properties of neutrinos and the dynamics of high-energy astrophysical processes.

A unique detection methods to identify tau neutrinos have been proposed, focusing on identifying the distinctive signatures, such as \emph{double bang} events that they possess—where a tau neutrino interaction produces a tau lepton that subsequently decays, generating additional light deposition \sidecite{double_bang}, generating two \emph{cascades} of energy depositions. IceCube has pioneered reconstruction techniques to search for such double cascade events, which have yielded the first non-zero measurements of tau neutrinos \sidecite{Juliana_paper}, alongside the development of a double pulse method utilizing convolutional neural networks that has recently identified 7 astrophysical tau neutrino candidates in IceCube \sidecite{CNN_tau}. 

The work presented in this thesis builds from the work done for some earlier studies \sidecite{marcel_thesis, Juliana_paper} but with more number of data. With more years of data, there has been a better understanding of the detector, particularly of the South Pole ice that affects the reconstruction of double cascade events, this was tested vigourously and adapted in the analysis presented in this thesis. The analysis identified five double cascade events within 12 years of IceCube data. With this higher number of double cascade events, a tighter constraint on the $\nu_{\tau}$ fraction was expected, though this was not achieved. While the measured fraction is non-zero for all flavours ($\nu_e:\nu_{\mu}:\nu_{\tau}=0.19:0.43:0.38$), the best-fit value is compatible with all the known source scenario.  A detailed investigation uncovered various issues related to the robustness of the reconstruction method, specifically with the ternary particle classifier used to categorize events, being highly sensitive to even minor changes in reconstruction settings (ice models, exclusion of the optical modules etc). While the limits could potentially be improved by combining this sample with other high-statistics datasets that could constrain the other two flavor fractions, the lack of robustness suggests that further examination is needed to enhance classification before conducting a second analysis. The thesis work also included the production of simulations and a sensitivity study for the planned IceCube extension, the IceCube Gen2. Due to its larger size and higher detection threshold, the detector will be able to detect more tau neutrinos, making it sensitive to measure the flavour fraction more significantly.


The thesis is structured as follows: \refch{nu_theory} and \refch{nu_theory_sources} provide an introduction to neutrino physics from both particle physics and high-energy neutrino physics perspectives. \refch{nu_icecube} introduces the IceCube Neutrino Observatory, detailing its detection principles and the medium used to detect neutrinos, along with descriptions of what these events look like in the detector. \refch{nu_samples} focuses on the sample of high-energy neutrinos used for analysis, detailing simulation production, reconstruction methods, and the ternary classifier utilized to classify events into three morphologies for flavor measurements. \refch{analysis} presents the statistical methods employed in the analysis, discussing various components and parameters used in the fit, along with observable distributions and sensitivity assessments. \refch{HESE12} shall discuss the results of the analysis in chronological order, detailing the various stages and tests conducted to understand the results. Finally, \refch{gen2} explores the envisioned extension of IceCube through the IceCube-Gen2 detector and presents sensitivity analyses performed for tau neutrino searches and flavor measurements with this instrument. The thesis concludes with \refch{summary}, summarizing key findings and offering an outlook on future research directions.


