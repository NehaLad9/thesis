\chapter{Abstract}
Cosmic rays and high-energy neutrinos share a fundamental connection, as both are produced in extreme astrophysical environments where cosmic rays interact with matter or photons near their acceleration sites to produce neutrinos. The observable flavor composition of neutrinos on Earth offers critical insights into potential production mechanisms at these cosmic sources. Due to neutrino oscillations over cosmic baselines, tau neutrinos—which are not directly produced at the sources—are expected to appear by the time neutrinos reach Earth, making their detection essential for studies of astrophysical neutrino sources and flavor evolution. 

This thesis aims to measure the flavor composition of high-energy astrophysical neutrinos, with a particular focus on detecting tau neutrino events within a high-energy starting event sample collected by the IceCube Neutrino Observatory in Twelve years of its livetime. Using a ternary classifier, high energy starting events are reconstructed into three distinct morphologies—single cascade, double cascade, and track. Five events were identified as double cascades, including four previously unobserved, along with 64 single cascades and 28 starting tracks, consistent with expectations from a single power-law spectrum of neutrino. Significant improvements were implemented in both simulations and reconstruction methods compared to prior analyses, including corrections for cross-sections, tau lepton polarization, and updated neutrino weights. Additional updates incorporated the latest South Pole ice models to better account for anisotropic light propagation, that affects the purity of the double-cascade sample.

The fit results, based on these three event morphologies, yielded a flavor composition measurement of \( f_{\nu_e} : f_{\nu_{\mu}} : f_{\nu_{\tau}} = 0.19_{-0.15}^{+0.26} : 0.43_{-0.17}^{+0.27} : 0.38_{-0.24}^{+0.37} \) with 68\% uncertainties. The findings slightly favor a muon-damped flavor production scenario (0:1:0) at the sources, though alternative scenarios cannot be excluded with more than 1-sigma significance. While these results did not provide high-significance exclusion of specific source scenarios, the advancements in reconstruction and simulation methods highlighted important limitations in current techniques, particularly regarding double-cascade classifications. The work presented in this thesis, therefore, lays the groundwork for improving double-cascade reconstruction methods in future studies in IceCube. 

The thesis concludes with projections for the future IceCube-Gen2 observatory, to detect the high energy tau neutrinos and measure flavour composition of the neutrino spectrum. The work involved producing simulations to adapt the aforementioned ternary classifier to incorporate almost 8 times larger detector volume of the IceCube-Gen2 and multi-PMT modules that were approximated as a isotropic spherical sensor. The results indicate not only an excellent potential to exclude various source scenarios at the production sites, but also a possibilty to be able to resolve an energy dependent flavour composition, to study the evolution of source population across the diffuse neutrino spectrum.
