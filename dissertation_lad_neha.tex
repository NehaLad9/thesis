% Load the kaobook class
\documentclass[
	fontsize=10pt, % Base font size
	twoside=true, % Use different layouts for even and odd pages (in particular, if twoside=true, the margin column will be always on the outside)
	%open=any, % If twoside=true, uncomment this to force new chapters to start on any page, not only on right (odd) pages
	numbers=noenddot,
	secnumdepth=1, % How deep to number headings. Defaults to 1 (sections)
]{kaobook}
\setcounter{secnumdepth}{3} % number subsubsections
\setcounter{tocdepth}{3} % put subsubsections in ToC
\renewcommand{\marginlayout}{%
	\newgeometry{
		top=25mm,             % height of the top margin
		bottom=25mm,          % height of the bottom margin
		inner=25mm,           % width of the inner margin
		textwidth=100mm,        % width of the text
		marginparsep=8.2mm,     % width between text and margin
		marginparwidth=49.4mm,  % width of the margin
	}%
}

% Choose the language
\usepackage[english]{babel} % Load characters and hyphenation
\usepackage[english=british]{csquotes}	% English quotes


% Load packages for testing
\usepackage{blindtext}
%\usepackage{showframe} % Uncomment to show boxes around the text area, margin, header and footer
\usepackage{showlabels} % Uncomment to output the content of \label commands to the document where they are used
\usepackage{adjustbox}
% Load the bibliography package
\usepackage{kaobiblio}
\addbibresource{minimal.bib} % Bibliography file
\usepackage{todonotes}

% 
% \usepackage{xargs}                      % Use more than one optional parameter in a new commands
% \usepackage[pdftex,dvipsnames]{xcolor}  
% \usepackage[colorinlistoftodos,prependcaption,textsize=tiny]{todonotes}
% \newcommand{\unsure}[2][1=]{\todo[linecolor=red,backgroundcolor=red!25,bordercolor=red,#1]{#2}}
% \newcommand{\change}[2][1=]{\todo[linecolor=blue,backgroundcolor=blue!25,bordercolor=blue,#1]{#2}}
% \newcommand{\info}[2][1=]{\todo[linecolor=OliveGreen,backgroundcolor=OliveGreen!25,bordercolor=OliveGreen,#1]{#2}}
% \newcommand{\improvement}[2][1=]{\todo[linecolor=Plum,backgroundcolor=Plum!25,bordercolor=Plum,#1]{#2}}
% \newcommand{\thiswillnotshow}[2][1=]{\todo[disable,#1]{#2}}
%


% Load mathematical packages for theorems and related environments
\usepackage{kaotheorems}

% Load the package for hyperreferences
\usepackage{kaorefs}

\graphicspath{{images/}{./figures}} % Paths where images are looked for

\makeindex[columns=3, title=Alphabetical Index, intoc] % Make LaTeX produce the files required to compile the index


\begin{document}

%----------------------------------------------------------------------------------------
%	BOOK INFORMATION
%----------------------------------------------------------------------------------------

\KOMAoptions{twoside=false}
\begin{titlepage}
	\begin{center}
	\vspace*{0.6cm}
	
	\LARGE
	\textbf{Something about Taus and Flavour in Icecube}
	\large
	
	\vspace{0.8cm}
	
	\textbf{Dissertation}\\
	zur Erlangung des akademischen Grades\\
	doctor rerum naturalium \\ 
	(Dr. rer. nat.) \\
	
	\vspace{0.5cm}
	
	im Fach: Physik \\
	Spezialisierung: Experimentalphysik\\
	
	\vspace{0.5cm}
	
	eingereicht an der \\
	Mathematisch-Naturwissenschaftlichen Fakultät\\	
	der Humboldt-Universität zu Berlin\\
	
	\vspace{0.5cm}
	
	von\\
	\textbf{Neha Lad M. Sc}\\
%	\vspace{0.8cm}
	
	\vspace{0.5cm}
	%Präsidentin der Humboldt-Universität zu Berlin\\
	%Prof. Dr.-Ing. Dr. Sabine Kunst\\
	 
	\vspace{0.5cm}
	 
	%Dekan der Mathematisch-Naturwissenschaftlichen Fakultät\\
	%Prof. Dr. Elmar Kulke\\
	
	\vspace{0.5cm}
	Gutachter:\\
	Prof. Dr. Marek Kowalski, Humboldt-Universität zu Berlin\\
	%Prof. Dr. Walter Winter, Humboldt-Universität zu Berlin\\
	%Prof. Dr. Antoine Kouchner,  Université de Paris\\
	\vspace{0.5cm}
	Tag der mündlichen Prüfung: date, month 2024
	
	\end{center}
	\newpage
	
	\vspace*{8cm}
	
	\textbf{Copyright Notice}\\
	This book is released into the public domain using the CC-BY-4.0 code. 
	
	To view a copy of the CC-BY-4.0 code, visit: \\\url{https://creativecommons.org/licenses/by/4.0/}
	
	\medskip
	\textbf{Colophon} \\
	This document was typeset with the help of \href{https://sourceforge.net/projects/koma-script/}{\KOMAScript} and \href{https://www.latex-project.org/}{\LaTeX} using the open-source \href{https://github.com/fmarotta/kaobook/}{kaobook} template class.\\
	
	The source code of this thesis is available at:\\\url{https://github.com//kaobook}, \\while the scripts used to generate the plots are available at: \\\url{https://github.com/NehaLad9/thesis.git}\\
	
	\medskip
	
	\textbf{Publisher} \\
	First printed in 2024 by Humboldt Universität zu Berlin
	
	\newpage
	
	\vspace*{5.0cm}
	
%\afterpage{\blankpage}
	
\end{titlepage}
%----------------------------------------------------------------------------------------

\frontmatter % Denotes the start of the pre-document content, uses roman numerals


%----------------------------------------------------------------------------------------
%	PREFACE
%----------------------------------------------------------------------------------------

\chapter{Abstract}
Cosmic rays and high-energy neutrinos share a fundamental connection, as both are produced in extreme astrophysical environments where cosmic rays interact with matter or photons near their acceleration sites to produce neutrinos. The observable flavor composition of neutrinos on Earth offers critical insights into potential production mechanisms at these cosmic sources. Due to neutrino oscillations over cosmic baselines, tau neutrinos—which are not directly produced at the sources—are expected to appear by the time neutrinos reach Earth, making their detection essential for studies of astrophysical neutrino sources and flavor evolution. 

This thesis aims to measure the flavor composition of high-energy astrophysical neutrinos, with a particular focus on detecting tau neutrino events within a high-energy starting event sample collected by the IceCube Neutrino Observatory in Twelve years of its livetime. Using a ternary classifier, high energy starting events are reconstructed into three distinct morphologies—single cascade, double cascade, and track. Five events were identified as double cascades, including four previously unobserved, along with 64 single cascades and 28 starting tracks, consistent with expectations from a single power-law spectrum of neutrino. Significant improvements were implemented in both simulations and reconstruction methods compared to prior analyses, including corrections for cross-sections, tau lepton polarization, and updated neutrino weights. Additional updates incorporated the latest South Pole ice models to better account for anisotropic light propagation, that affects the purity of the double-cascade sample.

The fit results, based on these three event morphologies, yielded a flavor composition measurement of \( f_{\nu_e} : f_{\nu_{\mu}} : f_{\nu_{\tau}} = 0.19_{-0.15}^{+0.26} : 0.43_{-0.17}^{+0.27} : 0.38_{-0.24}^{+0.37} \) with 68\% uncertainties. The findings slightly favor a muon-damped flavor production scenario (0:1:0) at the sources, though alternative scenarios cannot be excluded with more than 1-sigma significance. While these results did not provide high-significance exclusion of specific source scenarios, the advancements in reconstruction and simulation methods highlighted important limitations in current techniques, particularly regarding double-cascade classifications. The work presented in this thesis, therefore, lays the groundwork for improving double-cascade reconstruction methods in future studies in IceCube. 

The thesis concludes with projections for the future IceCube-Gen2 observatory, to detect the high energy tau neutrinos and measure flavour composition of the neutrino spectrum. The work involved producing simulations to adapt the aforementioned ternary classifier to incorporate almost 8 times larger detector volume of the IceCube-Gen2 and multi-PMT modules that were approximated as a isotropic spherical sensor. The results indicate not only an excellent potential to exclude various source scenarios at the production sites, but also a possibilty to be able to resolve an energy dependent flavour composition, to study the evolution of source population across the diffuse neutrino spectrum.

\chapter{Zusammenfassung}
Im IceCube haben wir viele Neutrinos, von denen wir einige mit sehr hoher Energie auswählen, 
verbringen 1 Jahr mit ihnen, um sie in drei Geschmackskategorien einzuteilen. Ich vermute, dass wir auf diese Weise etwas darüber erfahren, woher sie kommen.
Ziemlich normales Zeug, ganz und gar nicht rassistisch.
\cleardoubleoddpage


%----------------------------------------------------------------------------------------
%	TABLE OF CONTENTS & LIST OF FIGURES/TABLES
%----------------------------------------------------------------------------------------

\begingroup % Local scope for the following commands

% Define the style for the TOC, LOF, and LOT
%\setstretch{1} % Uncomment to modify line spacing in the ToC
%\hypersetup{linkcolor=blue} % Uncomment to set the colour of links in the ToC
\setlength{\textheight}{230\vscale} % Manually adjust the height of the ToC pages

% Turn on compatibility mode for the etoc package
\etocstandarddisplaystyle % "toc display" as if etoc was not loaded
\etocstandardlines % "toc lines as if etoc was not loaded
\listoftodos[Notes]
\tableofcontents % Output the table of contents

\listoffigures % Output the list of figures

% Comment both of the following lines to have the LOF and the LOT on different pages
% \let\cleardoublepage\bigskip
% \let\clearpage\bigskip

\listoftables % Output the list of tables

\endgroup
\cleardoubleoddpage
%----------------------------------------------------------------------------------------
%	MAIN BODY
%----------------------------------------------------------------------------------------

\mainmatter % Denotes the start of the main document content, resets page numbering and uses arabic numbers
\setchapterstyle{kao} % Choose the default chapter heading style
\chapter{Introduction}
\labch{intro}
The study of high-energy cosmic phenomena has long been driven by the quest to understand the processes that produce extreme energies in the universe. High-energy astrophysical events provide insights into the
most energetic processes, from the cores of active galaxies to supernova explosions. Early observations of cosmic rays marked humanity's first detection of high-energy particles from beyond Earth, but interpreting
these signals remained a challenge due to the charged nature of cosmic rays, which makes their path susceptible to magnetic fields, masking their origins. This challenge initiated the search for alternative messengers (that were not just photons), giving rise to the field of \textbf{astroparticle physics}—a field that blends astronomy, particle physics, and astrophysics to probe the high-energy

The history of our exploration into high-energy astrophysics is marked by a series of pivotal moments and discoveries. In the early 20th century, the first hints of cosmic rays were detected by Victor Hess, who observed the elevation of radiation levels at high altitudes, indicating an extraterrestrial origin for cosmic rays \sidecite{HESS_Balloon}. As the field matured, the concept of astroparticle physics emerged, characterized by its multidisciplinary approach that blends astrophysics, particle physics, and cosmology. The pioneering experiment conducted by Ray Davis and his team at the Homestake Mine successfully detected neutrinos produced by nuclear fusion in the solar core, an achievement that not only confirmed theories of solar energy production but also initiated a new era in neutrino astronomy \sidecite{homestake}. This was further exemplified by the detection of neutrinos from the supernova SN1987A, which provided a direct observation of a stellar explosion and highlighted the importance of neutrinos as messengers of cosmic events \sidecite{SN1987A_superK,SN1987A_Baksan,SN1987A_IMB}. 

The historical narrative of neutrinos is fascinating, tracing back to their theoretical postulation by Wolfgang Pauli in 1930, who sought to resolve the apparent violation of conservation laws in beta decay \sidecite{Pauli}. Following their prediction, neutrinos remained elusive until their experimental detection in the 1950s by Clyde Cowan and Frederick Reines \sidecite{nu_discovery}. Over the subsequent decades, significant strides were made in understanding neutrino properties, including the revelation of their oscillation phenomena \sidecite{Pontecorvo}, which indicated that neutrinos have mass—a groundbreaking finding that necessitated a revision of the Standard Model of particle physics. The detection of solar neutrinos not only validated the models of nuclear fusion in stars \sidecite{Bahcall} but also introduced the concept of flavor oscillations, shedding light on the relationships between different neutrino types and their behavior as they propagate through space \sidecite{Ahmad_2001}.

The emergence of multi-messenger astronomy has transformed our understanding of the universe, allowing us to draw connections between cosmic rays, gamma rays, neutrinos, and gravitational waves. This approach is particularly significant in light of monumental observations such as the detection of gravitational waves from binary black hole mergers by LIGO \sidecite[-1cm]{LIGO_mergerpaper} and the identification of high-energy neutrinos from a blazer by the IceCube Neutrino Observatory \sidecite{txspaper}. Each messenger complements the others, providing a more nuanced picture of high-energy phenomena. For instance, the detection of neutrinos from astrophysical sources, such as blazars and starbust galaxy, has underscored the potential for neutrino astronomy to elucidate the processes that generate cosmic rays and gamma rays. In this context, the contributions of neutrino astronomy are invaluable, as they allow us to probe the inner workings of some of the most energetic events in the universe.

The evolution of our understanding of neutrinos parallels significant advancements in detector technology, reflecting the challenges and triumphs of the field. Neutrinos are unique messengers due to their weak interactions with matter, making them exceedingly difficult to detect. However, this very property also makes them powerful tools for studying fundamental physics, as they can escape dense astrophysical environments that would otherwise obscure other forms of radiation. Over the years, innovative detection methods have emerged, such as the development of large-scale neutrino telescopes that utilize the Cherenkov radiation produced when neutrinos interact with matter. The conceptual introduction of neutrino telescopes began with early experiments like DUMAND \sidecite[-7cm]{PhysRevD.42.3613} and AMANDA \sidecite[-6cm]{ANDRES20001}, leading to the realization of the IceCube Neutrino Observatory at the South Pole \sidecite[-5cm]{Halzen:2010yj}. As a result of these advancements, the capabilities for neutrino detection have increased dramatically, allowing for the exploration of a broader range of astrophysical phenomena.

\textbf{The IceCube Neutrino Observatory} has achieved remarkable milestones since it started to operate fully since 2011, establishing itself as one of the most remarkable neutrino telescopes in the field of neutrino astronomy. Among its significant accomplishments is the discovery of astrophysical neutrinos, which marked a turning point in the field, confirming that high-energy neutrinos originate from cosmic sources rather than being solely produced in Earth's atmosphere \sidecite[-6cm]{Evidence_paper}. The observation of a correlation between high-energy neutrinos from the TXS 0506+056 blazar and its gamma-ray emission demonstrated the connection between different messenger particles \cite{txspaper}. Additionally, IceCube has made important contributions to the study of neutrinos from various sources, such as the sayfert galaxy NGC 1068 \sidecite[-7cm]{ngc1068} and also neutrinos from our own galaxy, Milky Way \sidecite[-6cm]{icecube_milkyway}. In addition to these sources, and discovery of the diffuse neutrino flux, IceCube has also been able to measure the features of the neutrino flux with striking accuracies via different event samples, that target various neutirno flavours \sidecite[-6cm]{cscd_6yr,diffusenumu,HESE7_sample, ESTES}. The most recent results, combining some of these data samples and another independent sample revealed a spectral break in the diffuse neutrino spectrum with >$4\sigma$ confidence \sidecite[-1.5cm]{globalfit_icrc, MESE_ICRC}. Such features in the spectrum, can help recognising what are the dominant sources at different energy ranges. Furthermore, IceCube has placed tight constraints on neutrino oscillation parameters by measuring atmospheric tau neutrino appearance \sidecite{IceCube_atm_numixing,IceCube:2024xjj} contributing to our understanding of fundamental particle interactions.  

In addition to IceCube, there also exists a water-based neutrino telescope in Mediterranean sea at present \textbf{the Astronomy With A Neutrino Telescope And Abyss Environmental Research experiment (ANTARES)} \sidecite{AGERON201111} (now decomissioned) and its sucessor (under construction) \textbf{A kilometercube Neutrino telescope (KM3NeT)} \sidecite{MARGIOTTA201483}. The IceCube detector has also proposed for its successor, called \textbf{The IceCube Gen2}, which will be about 8 times bigger than its current volume and will also host a radio detector array, to measure very high energy cosmic neutrinos via \emph{radio detection methods} \sidecite{Gen2_TDR}. These radio techniques to detect high energy neutrino has been used in experiments like \textbf{The Radio Neutrino Observatory Greenland (RNO-G)} \sidecite{rnog}, which benefitted greatly from other radio neutrino experments such as \textbf{the Askaryan Radio Array (ARA)} \sidecite{ARA} and \textbf{the Antarctic Ross Ice Shelf Antenna Neutrino Array (ARIANNA)} \sidecite{ARIANNA} experiments. These experiments utilize the Askaryan effect as detection principle, \sidecite{Askaryan}, that will significantly expand the energy range of neutrino observatories, unlocking new insights into the world of high energy astroparticle physics.

While the achievements of IceCube are commendable, the study of tau neutrinos—one of the three flavors of neutrinos—has garnered particular attention due to their limited appearances at high energies. Tau neutrinos were postulated as a theoretical necessity with the discovery of the tau lepton in 1975 \sidecite{PhysRevLett.35.1489,PERL1977487}, yet they remained undetected for decades, until 2001 by DONUT collaboration \sidecite{DONUT}. Their only recent observations raise significant questions about the sources of neutrinos. Unlike their electron and muon counterparts, tau neutrinos do not get produced in high energy sources or in atmospheric interactions of cosmic rays or terrestrial sources\footnote{aside from those produced from charmed mesons in cosmic ray interactions.}. Although neutrino oscillations guarantees a non-zero tau neutrino fluxes. Depending on matter and magnetic field environments at the particle accelerator sites, the neutrino production may happen with different fractions \sidecite{cite170,cite168}, meaning the most straightforward scenario where due to decays of pions and subsequent decays of produced muons and electrons can produce neutrinos with flavour ratios 1:2:0 (or $\frac{1}{3}:\frac{2}{3}:0$), which due to oscillations gets converted into $\frac{1}{3}:\frac{1}{3}:\frac{1}{3}$ on earth\footnote{assuming three flavour of neutrinos}. The ability to detect tau neutrinos is hence critical for measuring both the flavor composition of astrophysical neutrinos at high energies and the oscillation parameters at lower energies. Understanding tau neutrinos opens new avenues in exploring the fundamental properties of neutrinos and the dynamics of high-energy astrophysical processes.

A unique detection methods to identify tau neutrinos have been proposed, focusing on identifying the distinctive signatures, such as \emph{double bang} events that they possess—where a tau neutrino interaction produces a tau lepton that subsequently decays, generating additional light deposition \sidecite{double_bang}, generating two \emph{cascades} of energy depositions. IceCube has pioneered reconstruction techniques to search for such double cascade events, which have yielded the first non-zero measurements of tau neutrinos \sidecite{Juliana_paper}, alongside the development of a double pulse method utilizing convolutional neural networks that has recently identified 7 astrophysical tau neutrino candidates in IceCube \sidecite{CNN_tau}. 

The work presented in this thesis builds from the work done for some earlier studies \sidecite{marcel_thesis, Juliana_paper} but with more number of data. With more years of data, there has been a better understanding of the detector, particularly of the South Pole ice that affects the reconstruction of double cascade events, this was tested vigourously and adapted in the analysis presented in this thesis. The analysis identified five double cascade events within 12 years of IceCube data. With this higher number of double cascade events, a tighter constraint on the $\nu_{\tau}$ fraction was expected, though this was not achieved. While the measured fraction is non-zero for all flavours ($\nu_e:\nu_{\mu}:\nu_{\tau}=0.19:0.43:0.38$), the best-fit value is compatible with all the known source scenario.  A detailed investigation uncovered various issues related to the robustness of the reconstruction method, specifically with the ternary particle classifier used to categorize events, being highly sensitive to even minor changes in reconstruction settings (ice models, exclusion of the optical modules etc). While the limits could potentially be improved by combining this sample with other high-statistics datasets that could constrain the other two flavor fractions, the lack of robustness suggests that further examination is needed to enhance classification before conducting a second analysis. The thesis work also included the production of simulations and a sensitivity study for the planned IceCube extension, the IceCube Gen2. Due to its larger size and higher detection threshold, the detector will be able to detect more tau neutrinos, making it sensitive to measure the flavour fraction more significantly.


The thesis is structured as follows: \refch{nu_theory} and \refch{nu_theory_sources} provide an introduction to neutrino physics from both particle physics and high-energy neutrino physics perspectives. \refch{nu_icecube} introduces the IceCube Neutrino Observatory, detailing its detection principles and the medium used to detect neutrinos, along with descriptions of what these events look like in the detector. \refch{nu_samples} focuses on the sample of high-energy neutrinos used for analysis, detailing simulation production, reconstruction methods, and the ternary classifier utilized to classify events into three morphologies for flavor measurements. \refch{analysis} presents the statistical methods employed in the analysis, discussing various components and parameters used in the fit, along with observable distributions and sensitivity assessments. \refch{HESE12} shall discuss the results of the analysis in chronological order, detailing the various stages and tests conducted to understand the results. Finally, \refch{gen2} explores the envisioned extension of IceCube through the IceCube-Gen2 detector and presents sensitivity analyses performed for tau neutrino searches and flavor measurements with this instrument. The thesis concludes with \refch{summary}, summarizing key findings and offering an outlook on future research directions.





%%%%%%%%%%%%%%%%%%%%%%%%
%%%%%Intro Chapters%%%%%
%%%%%%%%%%%%%%%%%%%%%%%%
\pagelayout{wide} % No margins
\addpart{A Brief Introduction of Neutrinos in AstroParticle Physics}
\pagelayout{margin} % Restore margins
\chapter{Neutrinos in High Energy Universe}
\labch{nu_theory}

\section{Fundamental Properties of Neutrinos}
\label{sec:nu_properties}
\subsection{Masses and oscillations}

\subsection{Standard Model Interactions}

\subsection{Properties beyond standard model}



\section{Cosmic Rays}

\label{sec:cosmic_rays}

\subsection{Sources}

\subsection{Air Showers}


\section{Cosmic Neutrinos}
\label{sec:cosmic_nu}

\subsection{Sources and Production Mechanisms}

\subsection{Diffuse Fluxes}

\subsection{Flavour Composition}




\setchapterpreamble[u]{\margintoc}
\chapter{Neutrinos in IceCube} \labch{nu_icecube}

% Ever since the discovery of neutrino by Cowans and Reines \sidecite{nu_discovery},significant advancements have been made in detector technologies.
% These include the radiochemical detection of MeV-scale neutrinos produced by nuclear fusion in the Sun \sidecite{solarneutrinos}, and the detection of higher energy atmospheric neutrinos in deep underground telescopes \sidecite{ACHAR1965196,PhysRevLett.15.429}.
% The observation of neutrinos from supernova 1987a \sidecite{SN1987A_Baksan,SN1987A_superK,SN1987A_IMB} marked a significant milestone in astronomical research as it was the first-ever observation of such an event through these messengers, thereby opening a new window in the field of astronomy. \par

The birth of neutrino astronomy has led to the development of specific neutrino telescopes designed to explore the cosmos using these elusive particles. 
This chapter introduces the IceCube Neutrino Observatory, an advanced detector whose data forms the basis of the analysis discussed in this thesis. 
Later sections will explore different types of neutrino interactions and explain how experiments like IceCube can be used to detect the particle products of these interactions.\par 

\section{Detection of Neutrinos} 
\label{sec:nu_detection}

\subsection{Neutrino-Nucleaon Deep Inelastic Scattering}
\subsection{Energy losses of Particles in ice}
\subsection{Cherenkov Radiations}



\section{IceCube Neutrino Observatory}
\label{IC_detector}
As described in \ref{sec:nu_detection},the detection of high-energy neutrinos requires a large detector due to their small interaction cross-section. When these neutrinos interact, they produce Cherenkov photons which are produced due to the passage of charged daughter particles. Therefore, the detector must be transparent to these photons. Such a large detector volume can be acquired by using natural resources such as large bodies of water or ice; by deploying photosensors underneath to create a sufficiently sized detector.\par
This concept was first introduced in 1960 \sidecite{Markov:1960vja}. The groundwork for implementing such a detector began with water-based experiments like DUMAND \sidecite{PhysRevD.42.3613}, which was planned to be deployed in the sea near the main island of Hawaii and another detector with a similar design Lake Baikal \sidecite{BELOLAPTIKOV1997263}. First ever large-scale neutrino telescope built was predecessor of IceCube experiment called AMANDA \sidecite{ANDRES20001} at the geographic South Pole. A few hundred optical modules were dropped under the ice sheet of this dry continent between the depth of 1.5 to 2 km. Needless to say, the IceCube detector, the largest neutrino telescope in the world today, benefitted greatly in terms of design and performance from all the research and development work that was done with AMANDA. 
There also exists a large volume water-based neutrino telescope in the Northern Hemisphere called ANTARES\sidecite{AGERON201111}, and its successor KM3NeT\sidecite{MARGIOTTA201483}, located in the Mediterranean Sea.\par
The following subsections will discuss various detector and hardware components of the IceCube detector. Additionally, the last section will cover the optical properties of the South Pole ice, as these properties strongly affect the analysis observable and therefore influence the flavor measurements presented in this thesis.\par

\subsection{Detector}
\begin{figure}
	\centering \includegraphics{./figures/nu_in_icecube/IceCubeArray_slim.png}
	\caption{A schematic overview of the IceCube detector and its components \cite{Aartsen_2017}}
    \labfig{ic_detector}
\end{figure}
    
\begin{marginfigure}
	\includegraphics{./figures/nu_in_icecube/IC_Phase_Array.png}
	\caption{Top view of the location of each \emph{in-ice} strings of IceCube. Colour represents set of strings deployed in each seasons as described in \reftab{deployment_phases}. Note: IceTop Stations are not shown here.}
	\labfig{ic_phase_array}
\end{marginfigure}

\textbf{The IceCube Neutrino Observatory} is located at Amundsen-Scott South Pole Station at the geographic South Pole. It comprises a cubic kilometer of instrumented ice, equipped with 5,160 digital optical modules reffered as \emph{DOMs} from here-on (see \ref{sec:dom}), buried deep in the ice and 81 IceTop Stations on the surface of the ice, making itself the largest Neutrino Observatory in the world \sidecite{Aartsen_2017}. A schematic of the detector layout is shown in \reffig{ic_detector}. Four main components of the detector are \emph{in-ice array,DeepCore (inner extension of in-ice array),IceTop and IceCube Lab}. \par

\begin{description}
	\item[The Main \emph{in-ice} array] consists of 78 strings,\marginnote
    {\begin{kaobox}[title=\textbf{\emph{string} in IceCube}]
        An arrangement of DOMs attached on a twisted copper wire cable makes the so-called \emph{string} in IceCube.
    \end{kaobox}}
    each consisting of 60 DOMs, spaced vertically at a distance of 17 m between the depth of 1450 m to 2450 m under the ice sheet of Antarctica. Horizontal spacing between each of these strings is 125 m.  
	\item[\emph{DeepCore}] comprises inner 8 strings (see \reffig{ic_phase_array}) of the main\emph{in-ice} array, placed more closely together with horizontal string distance of 70 m and vertical DOM distance of 7 m \sidecite{deepcoredesign} between 1750 m to 2450 m depth. There's a region between depth of 1850 m to 2100 m, with no DOMs attached to the string as this region is the so-called \emph{dust layer} (see section \ref{sec:icemodel}), where optical scattering and absorprion is quite high and thus is not efficient to make reliable physics measurements. The Photomultipliers used in DOMs attached on these 7 strings also have higher Quantum Efficiency, which reduces the energy threshold to ~10 GeV. Although, IceCube's main goal is to detect astrophysical neutrinos, current topics of research spans much broader range (e.g fundamental properties of the neutrinos, such as oscillations \sidecite{Aartsen_2019_oscillations}, Physics beyond Standard Model searches such as Dark Matter \sidecite{Abbasi_2022} etc). 
	\item[\emph{IceTop}] is the surface detector array of the IceCube detector, primarily designed to detect Cosmic-ray airshowers and to be used as a veto layer for downgoing muons produced in these airshowers \sidecite{ABBASI2013188}. It consists of 81 \emph{stations} each having 2 tanks filled with clear ice (162 tanks in total). Each of these 162 tanks have 2 DOMs, similar to the ones deployed in \emph{in-ice} array, which makes it easier for both arrays to have a similar trigger and data acquistion system.
	\item[\emph{IceCube Lab}(ICL)] serves as the central operations' hub for the experiment, providing a crucial support for data acquisition and filtering. All the string cables connected to the aforementioned detcetor components are routed up to the ICL, from where triggered data is sent back to the northern hemispher via a satellite. Various other operations such as mainting the detector operations etc are also maintained from this building. 
	\end{description}

For the work presented in this thesis, neither \emph{deepcore} nor \emph{IceTop} data is used.
\begin{table}
    \caption{IceCube detector components deployed in each season (cumulative). Each configuration is represented as \emph{ICXX} where \emph{IC} stands for IceCube and \emph{XX} stands for number of total \emph{in-ice} strings at the end of that season}
    \labtab{deployment_phases}
    \begin{tabular}{cccc}
        \hline
        \hline
        Season & Configuration & Strings & IceTop Stations\\
        \hline
        2004-05 & IC1 & 1 & 4\\ 
        2005-06 & IC9 & 9 & 16\\ 
        2006-07 & IC22 & 22 & 26\\ 
        2007-08 & IC40 & 40 & 40\\ 
        2008-09 & IC59 & 59 & 59\\ 
        2009-10 & IC79 & 79 & 73\\ 
        2010-11 & IC86 & 86 & 81\\  
        \hline
        \hline
    \end{tabular}
    \end{table}


\marginnote{
    \begin{kaobox}[title=IceCube coordinates system]
        The IceCube coordinate system's origin is at 46500'E, 52200'N, 883.9 m elevation, which is quite close to centre of the \emph{in-ice} array. The y-axis points Grid North (toward Greenwich, UK), the x-axis points Grid East (90 degrees clockwise from North), and the z-axis points "up", forming a right-handed coordinate system.
    \end{kaobox}
    } 
The construction of the detector started taking place in 2005 and lasted for 7 Antarctic summer seasons till 2011 December. In each season, parts of the current day Hexagonal detector were deployed, as shown in \reffig{ic_phase_array} and numbers detailed in \reftab{deployment_phases}. A hot water drill was used to unfreeze the ice upto 2.5 km depth and 60 cm diameter into which these strings were then deployed (see \reffig{DOM_assembly}). The geometry of the detcetor, i.e The \emph{xy}-coordinates of the string were calculated from the drill tower position, surveyed during the deployment. Assuming the string was vertical, these coordinates were applied at all depths, with deviations of less than 1 m (later validated using the flasher data, see \ref{sec:DAQ}). Depths of the lowest DOM were determined from pressure readings, corrected for water compressibility and ambient air pressure, with vertical DOM spacings measured via laser ranger. All depths were converted to z-coordinates in the IceCube coordinate system \todo{Maybe draw a small schematic of coordinate system here next to the side note?}.\par  


\subsection{The Digital Optical Module (DOM)}
\label{sec:dom}
\textbf{The Digital Optical Module} (DOM) is a crucial component of the IceCube Neutrino Observatory, functioning as the heart of the detector. Each DOM is responsible for collecting the faint light signals produced by neutrino interactions in the Antarctic ice, amplifying these signals, and transmitting the data to the IceCube Lab \cite{Aartsen_2017}. From there, the data is relayed to the Northern Hemisphere via a satellite for further analysis. 

Inside each DOM, a 10-inch Photo Multiplier Tube (PMT) is positioned at the bottom, accompanied by essential circuitry for power conversion, data acquisition, calibration, control, and data transfer. Individual components of a DOM are as shown in \reffig{DOM_schematic}, functions of each of which is explained briefly below.
\begin{marginfigure}
    %\vspace*{-4.5cm}
    \includegraphics{./figures/nu_in_icecube/domfig2a-CableAssembly.pdf}
    \caption{A schematic of DOM CableAssembly being deployed in a water hole, created by hot water drill \cite{Aartsen_2017}.}
    \labfig{DOM_assembly}
\end{marginfigure}

\begin{marginfigure}
	\includegraphics{./figures/nu_in_icecube/domfig1a-DOM3DModel.pdf}
	\caption{A schematic of the DOM, showing its main components \cite{Aartsen_2017}.}
	\labfig{DOM_schematic}
\end{marginfigure}

\begin{description}
    \item[Glass Vessel Properties :] The glass vessel of the DOM is engineered to withstand the extreme pressures found in the deep Antarctic ice. This includes the constant long-term pressure of about 250 bar and the temporary pressure spikes of up to 690 bar experienced during the refreezing process after deployment using a hot water drill. The vessel is composed of two 0.5-inch thick hollow glass hemispheres, which are joined together with optical glue. This design not only provides a robust and hermetic seal to protect the internal electronics but also maintains the optical clarity necessary for the PMT to function effectively. The glass material is chosen for its strength, transparency, and resistance to the harsh conditions in the ice.

    \item[PMT :] The PMT within the DOM is a 10-inch diameter tube that utilizes a box-and-line dynode chain with 10 stages to amplify the faint light signals detected in the ice. The PMTs used in standard in-ice DOMs have a quantum efficiency peaking at 25\% whereas, DeepCore DOMs, designed to detect lower energy neutrinos, feature PMTs with a higher peak quantum efficiency of 34\% near the 390 nm wavelength. These PMTs are operated at a gain of $10^7$.

    \item[Gel :] A high strength, silicon gel is used between the photocathode area and the glass vessel to provide optical coupling and strong mechanical support to the DOM system. This gel has a high optical clarity, with 97\% transmission at 400 nm. It shows no signs of deterioration even after a decade, ensuring reliable performance.

    \item[Magnetic Shield :] The ambient magnetic field at the South Pole, measuring around 550 milligauss (mG), angled 17 degrees from vertical, can significantly affect the performance of the PMT. This includes reducing collection efficiency by 5-10\% and causing gain variations of up to 20\%, depending on the PMT's azimuthal orientation. To mitigate these effects, a mu-metal cage is installed around the PMT bulb, extending up to its neck. This cage is constructed from a mesh of 1 mm diameter wires with a spacing of 66 mm. Although this mesh blocks about 4\% of the incident light, it substantially reduces the adverse impacts of the magnetic field, ensuring more consistent and reliable performance of the PMT.

    \item[PMT Base and High Voltage Boards :] The high voltage board includes a Digital to Analog Converter (DAC) and an Analog to Digital Converter (ADC) for precise control and monitoring of the voltages supplied to the PMT. The high voltage generator on this board provides the necessary power, which is then regulated and distributed by the voltage divider circuits on the PMT base board. These circuits are specifically designed for low power consumption, ensuring efficient and stable operation of the PMT.
    
    \item[Main Board :] The main board serves as the central processing unit (CPU) of the DOM, managing and coordinating all other electronic components. It digitizes the waveforms detected by the PMT, providing a digital representation of the light signals for further analysis. The main board also temporarily stores data, calibrates the internal clock, and exchanges local coincidence information with neighboring DOMs (see \ref{sec:DAQ}). It communicates directly with the Data Acquisition (DAQ) system, ensuring the seamless transfer of data to the IceCube Lab. Additionally, the main board hosts an adjustable low-intensity optical source, which is used to calibrate the PMT's gain and timing, ensuring consistent and accurate performance.
    
    \item[Flasher Board :] Flasher board contains 12 LEDs each having specified output wavelength of $405\pm5 \mathrm{nm}$,
    %  \marginnote{
    %     \begin{kaobox}[title="color DOMs" or CDOMs]
    %         16 of the 5160 \emph{in-ice} DOMs have multi-wavelength LEDs on their Flasher boards. 8 of these DOMs are on string 79 (quite at the centre of the detector, one of the DeepCore strings) and the other 8 on string 14 on the edge of the detector. 
    %     \end{kaobox}}
        which generates lights \emph{in situ} to make various caliberation related measurements. In addition, this board can verify timing responses (useful for many, including analysis presented in this thesis, reconstruction processes). Additionaly, to measure \emph{the optical properties} of the South Pole Ice (see section \ref{sec:icemodel}) and locations of the DOMs in ice.
    
\end{description}

\subsection{Trigger and Data Acquisition}
\label{sec:DAQ}
Ever Since the initial deployment of the first string, the detector has been consistently gathering data, maintaining an average uptime of nearly 99\% \sidecite{Abbasi_2009}. \emph{The photocathode} of the PMT of a DOM, \emph{captures} a photon which then generates \emph{photoelectrons} which are accelerated through the series of 10 dynodes to generate measurable \emph{photocurrent}. This current is integrated over a time to obtain collected charge in units of \emph{photoelectrons or PEs}, through which \emph{photovoltage} is produced at the mainboard, over time, known as \emph{waveform}. These waveforms are then digitized to acquire and relay the data to the Northern Hemisphere.

Depending on how many photons hit the PMT, these waveforms can have different amplitudes ranging from 1mV upto the linearity limit of the PMT (~2 V) in time range of 12-1500 ns. The In Order to access this rather broad dynamic range, the digitiser used, Analog Transient Waveform Digitiser (ATWD) have three channels to amplify the waveform by factor of 0.25, 2 and 16. Moreover, 2 sets of ATWD are used that can operate alternatively in order to reduce the deadtime. ATWD can digitize voltage withinn duration of 427 ns, a window sufficient to reconstruct light produced within 10s of m around a given DOM. Naturally, some photons produced in energetic interactions may travel larger distances, producing faint but detectable waveforms. To amplify and digitize these waveforms, a fast Analog to Digital Converter (fADC) is also used, together ATWD+fADC is reffered as a \emph{DOMLaunch}.

The aforementioned digitization only happens if the voltage threshold of the onboard discriminator is met, which is kept at voltage equivalent to a PE of 0.25, or in other words, a DOM is \emph{hit}. If at least two neighbouring DOMs, on the same string produces individual \emph{hits} within 1 $\mu\mathrm{s}$ called \emph{Hard Local Coincidence(HLC)}, the full \emph{DOMLaunch} is transmitted to the surface. Otherwise, only the timestamp and minimal amplitude/charge information is sent known as \emph{Soft Local Coincidence(SLC)}. The HLC condition helps reduce false triggers from PMT dark noise, which is independent across DOMs. \emph{The Data Acquisition System (DAQ)} processes further uses these HLC hits to look for temporal coincidences. Most commonly used trigger in IceCube is the so-called \emph{The Single Multiplicity Trigger (SMT-8)}, that requires eight or more HLC hits within 5 $\mu\mathrm{s}$ timewindow. If and when SMT8 trigger conditions are met, all launches (HLC and SLC) are combined into what is called an \emph{event}. 

Various algorithms are used to make an estimate of event properties such as direction, deposited energy, morphology etc. The South Pole has limited computational resources, so only simple first guess algorithms can be used there called \emph{online filters}. Processed data is transmitted to the Northern Hemisphere. After this, more sophisticated reconstruction algorithms are applied to reduce and tailor the data as per Physics analysis goal. \todo{cite simulation/reco chapter here}


\section{Optical Properties of the South Pole ice}
The ice at the South Pole is a crucial part of the IceCube detector. It acts as both the detection target and the medium through which light propagates. Unlike the DOM hardware, which has been extensively studied in the laboratory, the glacial ice can only be measured in its actual environment, making it more challenging to describe. Understanding and calibrating this medium well is essential for accurate physics measurements,as it affects the systematic uncertainties in the measurements. The optical properties of the South Pole ice were first studied using ice cores from deep regions below 2000 m, collected from drill sites about 1000 km away from the South Pole. The deep South Pole ice was initially measured with the LED calibration system of AMANDA. Additionally, during the construction of IceCube, dust concentrations in the glacial ice were measured using a dust logger deployed into some drill holes.
\label{sec:icemodel}



%%%%%%%%%%%%%%%%%%%%%%%%
%%%Analysis Chapters %%%
%%%%%%%%%%%%%%%%%%%%%%%%
\pagelayout{wide} % No margins
\addpart{Tau neutrino identification and flavour composition analysis}
\pagelayout{margin} % Restore margins
\setchapterpreamble[u]{\margintoc}
\chapter{Event Sample,(Re)construction and Particle Identification}
\labch{nu_samples}

\section{Monte Carlo Simulation} 
\label{sec:mc_sim}

\subsection{Icecube simulation chain}
\label{sec:sim_ic}

\subsection{SnowStorm Simulation}
\label{sec:snowstorm}

\section{High Energy Starting Event (HESE) sample} 
\label{sec:HESE}

\section{Maximum Likelihood Event Reconstruction}
\label{sec:reco}

\section{Particle Identification of High Energy Neutrinos}
\label{sec:PID}

\subsection{Reclassification of PeV Double Cascades}
\label{sec:Pev_mask}

\subsection{Tau Polarisation}
\label{sec:Pev_mask}

\section{Influence of South Pole Ice properties on Double Cascades Reconstruction}
\label{sec:icemodel_checks}

\chapter{Flavour Composition Analysis}
\labch{analysis}


%%%%%%%%%%%%%%%%%%%%%%%%
%%%Result Chapters %%%
%%%%%%%%%%%%%%%%%%%%%%%%
\pagelayout{wide} % No margins
\addpart{Result}
\pagelayout{margin} % Restore margins
\setchapterpreamble[u]{\margintoc}
\chapter{Results}
\labch{HESE12}
The 7.5 years of HESE data (2010-2017) was previously used to measure the composition of astrophysical neutrino flavors \sidecite{Juliana_paper} (in particular to search for $\nu_{\tau}$ events) and energy spectrum \sidecite{HESE7_sample}. This dataset included 102 events, (of which 60 events were above 60 TeV), that passed the HESE selection criteria, as outlined in section \ref{sec:HESE}. Two events were identified as Double Cascade candidates using the particle identifier described in section \ref{sec:PID}. 

While the selection criteria and particle identification remain consistent, several key differences distinguish the analysis presented in this thesis from prior iterations. The most significant change lies in the ice model. As detailed in section \ref{sec:icmodel}, ice model properties—especially anisotropy—strongly influence the reconstruction of tau decay length. If not properly addressed, this can introduce bias in correctly tagging Double Cascade events. The ice model used in this analysis, SpiceBfr, can also impact the number of photons collected over time \sidecite{BFR_paper}, depending on the alignment of DOMs with respect to the iceflow axis. This, in turn, can affect overall energy estimates. Additionally, the treatment of detector systematics has changed; this analysis utilizes the SnowStorm method (see section \ref{sec:snowstorm}), whereas previous iterations relied on discrete Monte Carlo simulation sets. Another difference is also updated reconstruction tables used for maximum likelihood reconstruction method described in section \ref{sec:reco}. Other small corrections have also been applied to monte carlo simulations in terms of reweighting them to include corrections due to tau polarization and initial state radiation corrections to the Glashow cross-sections (see section \ref{sec:PID}). Other differences include variations in nuisance parameters and analysis software (see section \ref{sec:components}). Due to these changes, the 7.5 years of HESE data was re-unblinded as a first step.

This chapter presents flavor measurements made using 12 years of HESE data. It begins by discussing the re-unblinding of 7.5 years of data, followed by results from the 12-year fit, including Data-Monte Carlo agreement and detailed post-unblinding checks. Finally, the flavor measurement results are presented and interpreted in the last sectinons.

\section{(Re)Unblinding of 7.5 years of HESE Data}
\label{sec:HESE7}
The re-unblinding of the HESE-7.5 data provided new insights, revealing that 64 events met the HESE selection criteria, each with a deposited energy exceeding 60 TeV. It included 6 Double Cascade events, as summarized in Table 1, which is a significant increase from the previous analysis that identified only two Double Cascade events. Notably, 4 of the additional events had initially been classified as single cascades. The reclassification was largely driven by the application of the energy asymmetry cut, which proved to be a crucial factor in differentiating between single and double cascades (see section \ref{sec:PID}). Despite the changes in classification, the two common Double Cascade events identified in both iterations exhibited nearly identical reconstructed properties, as outlined in Table 2.

A key difference in this re-unblinding analysis was the inclusion of high quantum efficiency Digital Optical Modules (DOMs) from DeepCore. In prior analyses, these DOMs were excluded from the reconstruction of high-energy neutrino events, particularly in millipede-based reconstructions (detailed in Section 4). The reason for their exclusion was due to their smaller statistical uncertainties, compared to the larger systematic uncertainties associated with digitized waveforms. These systematic uncertainties, not being well-characterized for individual DOMs, could not be incorporated into the likelihood fitting process. However, advancements in simulation, including improved reconstruction tables and detector simulations, DeepCore DOMs were included for the analysis presneted in this thesis.

Following the observation of increased number of double cascade events, detailed checks were conducted to explore why Monte Carlo predictions had underestimated the number of Double Cascade events. Figure X shows the monte carlo distribution of Double Cascade classifications, both with and without the inclusion of DeepCore DOMs. The monte carlo simulations predicted only 2-3 Double Cascade events, yet the data revealed 6 Double Cascade events when DeepCore DOMs were included and only 3 when they were excluded. This discrepancy pointed to potential issues in either the simulation or reconstruction processes involving DeepCore DOMs. Additionally, the charge distribution of DeepCore DOMs relative to the total charge of these events uncovered another discrepancy: while the MC simulations remained consistent, the real data distribution differed significantly between the cases where DeepCore DOMs were included or excluded. This raised concerns about the accuracy of the simulation and the treatment of DeepCore DOMs, indicating that further investigation at the MC level is necessary.

Considering the historical exclusion of DeepCore DOMs (as well as other "bad" DOMs like bright or saturated modules) from previous reconstruction chains, this analysis ultimately decided not to include DeepCore DOMs in the full sample unblinding. Consequently, the re-unblinding of the HESE-7.5 data resulted in 62 events with deposited energies above 60 TeV. Of these, 45 were classified as single cascades, 3 as Double Cascades, and 14 as track events. A detailed comparison between these newly unblinded results and previous results, including morphologies and reconstructed properties, is presented in the accompanying tables.



\section{Fit results}
\label{sec:HESE12}

\section{Data/Monte Carlo Agreement}
\label{sec:data_mc}

\section{Flavour Composition of Diffuse Astrophysical Neutrinos}
\label{sec:flavour_results}

\section{Discussion}
\label{sec:results_discussion}
\chapter{GlobalFit Results}
\labch{globalfit}

\setchapterpreamble[u]{\margintoc}
\chapter{Sensitivity of IceCube-Gen2 to measure flavour composition of Astrophysical Neutrinos}
\labch{gen2}
This chapter details the sensitivity studies performed for \emph{The IceCube Gen2 detector}. A siginificant portion of this thesis work was dedicated to assess and derive sensitivity to measure the flavour composition of Astrophysical Neutrinos for IceCube Gen2. The detector will be introduced in the following sections, along with the simulations and software framework used to produce the results. 

\section{IceCube Gen2} 
\label{sec:gen2-detector}
IceCube-Gen2 is a proposed next generation of neutrino detector, designed to observe the neutrino sky within a wide energy range, from TeV to EeV \sidecite{whitepaper}. Its sensitivity is expected to be at least five times better than IceCube, enabling the observation of individual sources. The instrument layout is designed to detect about ten times more neutrinos annually as compared to IceCube. This increased capability will facilitate in-depth studies of neutrino distribution across the sky, energy spectrum, and flavor composition and beyond standard model physics.
\begin{figure}[h!]
	\includegraphics[scale=1.5]{./figures/gen2/decadal_survey_gen2-fan_radio_geometry.pdf}
	\caption{Figure depicts the proposed IceCube-Gen2 Neutrino Observatory facility at the South Pole. It includes (from left to right) (i) a radio array with 200 stations, (ii) 120 new in-ice strings, spaced 240 m apart (shown as orange points), as an expansion of (iii) current optical array, (iv) 7 strings of IceCube upgrade, to be deployed soon within currrent in-ice DeepCore volume. Figure taken from \cite{whitepaper}}
	\labfig{gen2_all_geometry}
\end{figure}

\reffig{gen2_all_geometry} illustrates a top view of the IceCube-Gen2 facility, showcasing its various components using optimized technologies for the targeted energy ranges. 

\begin{description}
    \item \textbf{\emph{The IceCube Upgrade}} will start deployment this season. Its goal is to lower the detection threshold for neutrinos to 1 GeV (In-line with its predecessor, \emph{DeepCore} in current IceCube)\sidecite{AYA}. This improvement will advance oscillation measurements, dark matter searches, and studies of physics beyond the Standard Model. The IceCube Upgrade project will also deploy 693 new types of multi-PMT detector modules, providing an opportunity to test the optical sensor technology for the IceCube-Gen2 observatory.

    \item \textbf{\emph{The surface array}} of IceCube-Gen2 is a unique setup where the surface array measures the electromagnetic shower component and low-energy muons, while the optical array detects TeV and potentially PeV muons from the same air shower \sidecite{IceCubeCollaborationSchroeder2024_1000168735}. Planned to be used similarly as \emph{IceTop} of IceCube, the stations shall be placed on top of the additional \emph{in-ice} strings of optical array. It can also be used as \emph{surface veto} to reduce the background of atmospheric muons in samples of astrophysical neutrinos from the southern sky.

    \item \textbf{\emph{The Radio array}} aims to discover and characterize high-energy neutrino flux above 10 PeV. It detects nanosecond-scale radio emissions from ultra-high-energy particle showers using the Askaryan effect \sidecite{Askaryan,meyers_21}. This technique is sensitive to energies above PeV and complements the energy range of the optical array by capturing radio emissions from neutral and charged-current interactions, as well as energy losses of secondary leptons. 

    \item \textbf{\emph{The optical array}} The optical array will be expanded with the addition of 120 new strings to the existing IceCube strings. The strings will be arranged in what is referred to as "sunflower geometry," with an average horizontal spacing of 240 meters. The shape of the array and spacing between the strings will be determined through dedicated geometry optimization studies. Each string will contain 80 modules, resulting in a total of 9600 new modules. These modules will be placed between 1325 meters and 2575 meters below the surface, with a vertical spacing of 16 meters. This configuration will create an instrumented geometric volume of 7.9 cubic kilometers. The modules on the string are expected to collect nearly three times the number of photons gathered by an IceCube digital optical module (DOM).\todo{cite TDR link here? or is whitepaper ok?}
\end{description}

For the sensitivity study presented in this thesis, only the optical part of the proposed detector was simulated and used. 

\begin{figure}
	\includegraphics[scale=2.2]{./figures/gen2/Gen2_DOMs.pdf}
	\caption{The designs of the IceCube-Gen2 optical sensors, DOM-16(second) and DOM-18 (third) with their base designs, to be used in the IceCube Upgrade sensors, are the mDOM on the left and D-Egg on the right \cite{Gen2_TDR}}
	\labfig{Gen2DOMs}
\end{figure}

\section{Simulation}
\label{sec:gen2-sim}
To perform this sensitivity study, dedicated simulations were carried out. The study aims not only to assess the sensitivity of IceCube-Gen2 in measuring the flavor composition of astrophysical neutrinos but also to evaluate its capabilities in detecting tau neutrino events, which is a crucial component as described in \ref{nu_samples}. The simulations were aligned with the mainline IceCube simulations (detailed in \ref{nu_samples} \todo{cite the simulation section here and not the chapter}) to enable direct comparisons. However, necessary modifications were made to account for the new-generation optical sensors to be used in IceCube Gen2 and the sparser geometry. The following sections will describe the event samples created using these simulations to conduct the sensitivity analysis.

\subsection{Isotropic Sensor}
\label{sec:isopdom}
The optical sensors to be used in the IceCube-Gen2 project depends a lot on how well the reference optical sensors to be deployed in the IceCube upgrade perform.\sidecite{Gen2_TDR}. The designs have been carefully optimized to balance cost-effectiveness, logistical efficiency, and enhanced performance. \reffig{Gen2DOMs} shows both the 16 and 18 PMT modules, which are being considered to use in IceCube Gen2, along with \textbf{mDOM} (\emph{multi PMT Digital Optical Module}) \sidecite{mDOM_2017,mDOM_2019} and \textbf{D-Egg} (\emph{Dual optical sensors in an Ellipsoid Glass for Gen2}) \sidecite{D-Egg_MainPaper} that are to be deployed in ice for IceCube Upgrade.

\marginnote{\begin{kaobox}[title=pDOM]
    pDOM stands for PINGU Digital Optical Module. It was first coined for an R\&D upgrade of IceCube DeepCore called \textbf{PINGU} (The Precision IceCube Next Generation Upgrade) \cite{PINGU}.
\end{kaobox}}

The maturity of the design, along with extensive in-situ testing using a large number of sensors for the IceCube Upgrade, leads us to consider the mDOM-type sensor as the baseline for evaluating the IceCube Gen2 detector's capabilities in identifying Tau neutrino-induced Double Cascade events. Unlike IceCube’s single large 10" PMT, the mDOM consists of 24 smaller 3" PMTs. The key advantages of the mDOM over pDOM \sidecite{PINGU} \todo{did this project eventually become Upgrade?} are its 2.2 times higher effective photocathode area, omnidirectional sensitivity, and the directional information obtained from the individual “pixels” (the 24 PMTs). Due to the large number of PMTs and their strategic placement within the module sphere, this module offers nearly isotropic angular acceptance, unlike IceCube DOMs with only one downward-facing PMT. 

The effective area of the optical modules is the equivalent physical cross-section that would detect all the incident photons from a plane perpendicular to a given direction. As illustrated in \reffig{EffectiveArea_mDOM} (Left plot) \todo{reproduce this figure, without the right plot}, the mDOM has a nearly linear effective area for collecting photons from all directions, unlike the Gen1 DOMs (pDOMs) which have a downward-facing PMT. As a result, the effective area for pDOMs increases as the arrival direction shifts from 180 degrees ("down-going" in the IceCube coordinate system) to 0 degrees ("up-going" in the IceCube coordinate system).

\begin{figure}
    \includegraphics{./figures/gen2/EffectiveAreaCurve_TDR.pdf}
    \caption{The effective area is compared for IceCube-Gen2 DOM candidates, 16, and 18 PMT models, in relation to IceCube-Gen1 DOM (pDOM), D-Egg, and mDOM, as functions of zenith angle (left) and wavelength averaged over solid angle (right). Figure taken from \cite{Gen2_TDR}} 
    \labfig{EffectiveArea_mDOM}
\end{figure}


However, current sophisticated methods do not yet provide a full-scale simulation of a multi-PMT module, so a simulated sensor called \emph{iso-pDOM} (isotropic-pDOM) was developed \sidecite{Anastasiia_Thesis}. This sensor can be thought of as a 'spherical PMT' encased in a glass vessel similar to an IceCube DOM but with 2.2 times higher quantum efficiency, capable of capturing photons arriving from all the directions (see \reffig{isoPDOM_schematic}). The sphere was simulated by assuming an upward-facing PMT along with a downward one and combining the results while maintaining the same area under the curves at all wavelengths \todo{cite Evangelia's Summer school report?}. The resultant iso-pDOM has an effective area very similar to that of an mDOM.

\begin{marginfigure}
    \includegraphics{./figures/gen2/iso-pDOM.pdf}
    \caption{Conceptual representation of Simulated sensor with isotropic angular acceptance (iso-pDOM)}
    \labfig{isoPDOM_schematic}
\end{marginfigure}

\begin{marginfigure}
    \includegraphics{./figures/gen2/isopDOM_eff_area.png}
    \caption{Results of simulating a sensor that \emph{mimics} the behaviour of a typical mDOM. The blue line shows changed effective area of the so-called \emph{isopDOM}, achieved by combining acceptance curves of pDOMs having a PMT in "upper" (orange) and "lower" (grey) halves of the DOM respectively}
    \labfig{isoPDOM_effarea}
\end{marginfigure}


\subsection{Event Selection}
\label{sec:gen2_eventsample}
\todo{many references from chapter 4 need to be cited correctly in this section}

Monte Carlo events were produced using the so-called isoPDOM for all three flavors of primary neutrinos with energies ranging from 100 TeV to 50 PeV. The simulation chain is identical to that described in \ref{nu_samples}. It is important to note that since the IceCube in-ice array is inherently part of the proposed Gen2 detector, all simulated events still include 'hits' from the IC86 configuration. Additionally, during the DetectorSim stage of the simulation chain, where PMT responses, noise, etc., are added, responses are incorporated separately for IceCube DOMs and isoPDOMs. If an event passes all the basic triggers, two separate triggers are stored depending on the event location: IC86 and ICGen2. By default, ICGen2 has a combined response of both detector configurations, while IC86 only contains current IceCube volume events. This feature is crucial as it facilitates direct comparison of events produced with IceCube simulations for IceCube-only analyses.

As discussed in analysis chapter, a fundamental aspect of flavor measurement studies is the ability to identify the flavor of the neutrino involved in an interaction. This identification is possible due to the distinct by-products produced by different neutrino interactions, which result in unique light deposition patterns, or "morphologies" reco chapter. These patterns, illustrated in cite morphology figures, allow us to reconstruct the events by analyzing the morphology of the light deposited in the detector. By doing so, the flavor of the original interacting neutrino can be determined.

To utilize the same particle identifier used in the analysis presented in analysis chapter for this sensitivity study, a dedicated event selection process for high-energy starting events (HESE) was implemented, similar to the approach described in reco chapter. However, since the outer-layer detector veto is specific to the detector geometry and the characteristics of the DOM pulses—which is still under development for the IceCube-Gen2 simulation chain, starting events were selected by examining the interaction vertex of the primary neutrino. This interaction vertex was further refined by considering the deposited charge (measured in single photoelectrons) and calculating the charge-weighted mean positions. This charge information is crucial for applying a HESE-like charge cut. Unlike the 6000 photoelectrons threshold used previously, the threshold for this analysis was set at 2000 photoelectrons. The lower threshold is due to the higher quantum efficiency and isotropic sensitivity of the new sensors, which enhance the detection capability for high-energy events. All the approximations made were in parallel checked for IC86 configuration to reproduce MonteCarlo PDFs within statstical erros to the ones presented in analysis chapter.

Moreover, to appropriately weight the simulated events and account for the probability of an atmospheric neutrino being rejected by an accompanying muon triggering the veto, a dedicated calculation similar to the one used in the HESE-7.5 analysis \sidecite{HESE7_sample} was used. Additionally, the reconstructed energy cut, initially set at 60 TeV, was adjusted to 100 TeV. This adjustment was based on the signal-to-background probability density functions (see \reffig{DC_signal_gen2} and \reffig{DC_bkg_gen2}) to ensure a similar signal-to-background ratio (2:1) as achieved in the HESE-7.5 analysis \sidecite{Juliana_paper} and the HESE-12 analysis (\todo{cite analysis chapter}). After applying all the necessary filters, the final sample includes starting events with a reconstructed deposit energy of 100 TeV or more and a charge exceeding 2000 PE. These events originate from interactions of all six types of neutrinos (particle and antiparticle versions of 3 flavors) beginning within the simulated IceCube-Gen2 fiducial volume. They are divided into three categories: Tracks, Single Cascades, and Double Cascades.

\begin{figure}
\centering
\begin{subfigure}{.8\textwidth}
      \centering
      \includegraphics{./figures/gen2/Signal_PDF.png}
      \caption{Double Cascades from $\nu_{\tau}$ interactions (signal). The signal double cascades show a correlation between ($\mathrm{L}_{\mathrm{dc}}$) and ($\mathrm{E}_{\mathrm{tot}}$).}
      \labfig{DC_signal_gen2}
\end{subfigure}
\begin{subfigure}{.8\textwidth}
      \centering
      \includegraphics{./figures/gen2/Bkg_PDF.png}
      \caption{Double Cascades from $\nu_{\mathrm{e}}$ and $\nu_{\mu}$ interactions (background). The background contributions do not show any correlations, but rather clusters at low $\mathrm{L}_{\mathrm{dc}}$ and $\mathrm{E}_{\mathrm{tot}}$.}
      \labfig{DC_bkg_gen2}
\end{subfigure}
\caption{Total reconstructed energy ($\mathrm{E}_{\mathrm{tot}}$) versus reconstructed double cascade length ($\mathrm{L}_{\mathrm{dc}}$) for events classified as double cascades.}
\labfig{DC_PDFs}
\end{figure}


\section{Analysis tool : \texttt{toise}}
\label{sec:toise}
Understanding and enhancing the sensitivity of the detector can result in more precise and dependable performance, thus improving the scientific impact of the experiment. The main goal of the sensitivity studies carried out in this work is to optimize the design of the detector in order to be capable of reconstructing tau neutrino events using existing methods but with a larger detection volume and new generation optical sensors. Additionally, one can also assess the performance of the detector against theoretical model \sidecite{muon_damp} by combining both radio and optical arrays of the proposed detector to investigate flavor measurements in the energy ranges from TeV to EeV \sidecite{Coleman}.

It's impractical to run comprehensive simulations for evolving detector designs due to the large amount of computing power required. The \texttt{toise} \sidecite{toise} framework was created to estimate sensitivity using a simplified model of the detector response based on targeted Monte Carlo (MC) simulations. This allows for efficient comparisons of different detector designs without repeating the entire simulation process. \texttt{toise} was used for a sensitivity study presented in the next section. This section will provide a brief overview of its workflow. In order to distinguish the influence of design choices on detector performance from the intrinsic restrictions imposed by neutrino interaction physics, the event rate calculation in this framework is conducted through two distinct stages: Neutrino Physics and Detection.

In the Physics stage, the neutrino fluxes at the Earth's surface are converted to the detector's area or volume. This involves using a transfer tensor to model the conversion between the initial neutrino flavor states and the observable final states (muons, hadrons, etc.). In addition, various aspects of neutrino interactions, including neutrino-nucleon cross-sections and different interaction types (neutral current or charged current) are also taken into account. The transfer tensor is subsequently combined with the final-state effective area to establish a neutrino effective area. The effective area \( A_{\text{eff}}(E, \theta) \) of the detector is calculated by multiplying the geometric area \( A_{\text{geo}}(\theta) \) with an energy and zenith-dependent efficiency \( \eta(E, \theta) \):

\begin{equation}
    A_{\text{eff}}(E, \theta) = A_{\text{geo}}(\theta) \times \eta(E, \theta)
\end{equation}

For the optical array of the proposed IceCube-Gen2, the geometric area is approximately calculated by placing a convex hull around the instrument's geometric boundary. \textbf{\emph{The selection efficiency}} \( \eta(E, \theta) \) characterizes the detector's triggering efficiency and the probability of an event passing a set of analysis criteria. It is defined as \emph{the ratio of events passing these cuts to the number of events generated}. Depending on the type of sensitivity study being performed—such as expected limits, discovery potential, or flavor measurement—additional parameterizations like energy and angular resolutions and classification efficiency are used. For flavor measurement, \textbf{\emph{the classification efficiency}} generates an event classification smearing matrix (\reffig{fig:classeff}) and is defined as \emph{the fraction of topology per energy bin for a given neutrino flavor}.
\todo{perhaps a block diagram/dummy figure to show the whole flow on the side?}
When estimating sensitivities, it is essential to account for backgrounds that may mimic the signal. The framework handles backgrounds by either adding their contributions to the event rate or ignoring regions where they are expected to contribute. For all detectors and science cases, atmospheric neutrino flux is added as a background using the same effective area as for astrophysical neutrinos. Optionally, atmospheric neutrino flux in the downward-going region can be reduced to account for vetoing by accompanying muons from the same air shower.

\todo{more discription will be there in analysis chapter for HESE12, if not, make this more detailed?}
 
\section{Result of Flavour Sensitivity Measurements}
\label{sec:gen2-results}
Using the HESE-like sample, described in \ref{sec:gen2_eventsample}, a detector response tensor is generated using \texttt{toise}. Selection efficiency, detailed in \ref{sec:toise}, is a key factor in generating the detector's neutrino effective area. This efficiency is determined using \reffig{fig:selecteff}, which shows the true deposited energy at which the analysis starts to select events. The plot illustrates the ratio of events passing all selection cuts in \ref{sec:gen2_eventsample} to all simulated neutrinos reaching the fiducial volume, per energy bin. The curve is fitted, and the resultant plot shows that neutrinos are selected starting from approximately 200 TeV. This value is used as the selection threshold for events beginning or contained in the fiducial volume.

\begin{figure}[h!]
    \centering
      \includegraphics{./figures/gen2/SelectionEff.pdf}
      \caption{Selection Efficiency : Ratio of neutrinos that got classified into a topology to all the neutrinos that interacted in active volume. Data here refers to monte-carlo events per energy bin.}
  \labfig{selecteff}
  \end{figure}

% A specific parameterized tensor is used to determine the efficiency of particle identification during flavor measurements. \reffig{classeff} shows how well the reconstruction process can identify different types of particles based on their shapes. In an ideal scenario, events involving charged current interactions with electron neutrinos ($\nu_e$) are classified as single cascades, those involving muon neutrinos (\nu_{\mu}) as tracks, and those involving tau neutrinos (\nu_{\tau}) as double cascades (all neutral current events appear as single cascades, as explained in \todo{chapter 4 section 2}). The diagonal elements of the plot show how accurate the classifier is, while the off-diagonal elements indicate the fractions of misidentified flavors. The plot shows that as the true deposited energy increases, the number of double cascade events (from \nu_{\tau} interactions) initially plateaus and then decreases. This occurs because at higher energies, the individual energy depositions are further apart (due to correlation of $\mathrm{L}_{\mathrm{dc}}$ and $\mathrm{E}_{\mathrm{tot}}$), making it easier for the reconstruction process to distinguish them apart. However, at even higher energies, one of the cascades may be partially or completely outside the detector, causing these events to be misclassified as single cascades due to strict containment criteria (\todo{see reco section}). For single cascades involving \nu_e, the efficiency decreases at high energies because some DOMs may become saturated, and their data is excluded from the analysis. In contrast, the efficiency for starting tracks remains relatively consistent across the entire energy range.

A specific parameterized tensor is used to determine the efficiency of particle identification during flavor measurements. \reffig{classeff} shows how well the reconstruction process can identify different types of particles based on their shapes. In an ideal scenario, events involving charged current interactions with electron neutrinos ($\nu_e$) are classified as single cascades, those involving muon neutrinos ($\nu_{\mu}$) as tracks, and those involving tau neutrinos ($\nu_{\tau}$) as double cascades (all neutral current events appear as single cascades, as explained in \todo{chapter 4 section 2}). The diagonal elements of the plot show how accurate the classifier is, while the off-diagonal elements indicate the fractions of misidentified flavors. The plot shows that as the true deposited energy increases, the number of double cascade events (from $\nu_{\tau}$ interactions) initially plateaus and then decreases. This occurs because at higher energies, the individual energy depositions are further apart (due to correlation of $\mathrm{L}_{\mathrm{dc}}$ and $\mathrm{E}_{\mathrm{tot}}$), making it easier for the reconstruction process to distinguish them apart. However, at even higher energies, one of the cascades may be partially or completely outside the detector, causing these events to be misclassified as single cascades due to strict containment criteria (\todo{see reco section}). For single cascades involving $\nu_e$, the efficiency decreases at high energies because some DOMs may become saturated, and their data is excluded from the analysis. In contrast, the efficiency for starting tracks remains relatively consistent across the entire energy range.


\begin{figure}[h!]

    \centering
    \includegraphics{./figures/gen2/ClassificationEff_small_watermark.pdf}
    \caption{Classification Efficiency: Three subplot columns are true neutrino flavors, where each energy bin (true Monte Carlo energy) contains the fraction of topologies, summing to 100\%. Diagonal plots show the flavor identification efficiency of the classifier, whereas off-diagonal plots show misidentification fractions.
    }
    \labfig{classeff}
\end{figure}

Lastly, a significant advantage of \texttt{toise} is its ability to combine different event selections and detector types. The starting event sample from this detailed study can be combined with the efficiencies of through-going tracks (see Chapter 4). In toise, these efficiencies are included by extrapolating IceCube analysis limits \sidecite{diffusenumu} to calculate angular resolutions, PSF, etc. The flavor measurement presented here demonstrates the sensitivity of IceCube Gen2 by combining starting events with through-going muons, a method already realized and updated in IceCube \sidecite{Neha_ICRC_IC}.

\reffig{Gen2_Flavortriangle} shows the projected flavor measurement sensitivity of IceCube-Gen2 with 10 years of data \sidecite{Neha_ICRC_Gen2}. The Asimov dataset assumes equal partition of all flavors, with a diffuse neutrino spectrum following a single power-law with an index of 2.5 and a per-flavor normalization of 2.3 \sidecite{lars_globalfit}. It is worth to note that systematic errors are excluded from this study. \reffig{Gen2_Flavortriangle_nonutau} illustrates the sensitivity change if no dedicated $\nu_{\tau}$ identifier is used in the starting event sample, resulting in the sample containing only single cascades and tracks, making it impossible to resolve the $\nu_e$ and $\nu_{\tau}$ fraction degeneracy.

\begin{figure}[h!]
    \centering
    \includegraphics{./figures/gen2/Gen2-10Years.pdf}
    \caption{Projected sensitivity of IceCube-Gen2 to measure flavor composition of Astrophysical neutrino with 10 years of its lifetime. The dashed (solid) outlines depict the corresponding 99\% (68\%) constraints.}
    \labfig{Gen2_Flavortriangle}
\end{figure}

The sensitivity shown in \reffig{Gen2_Flavortriangle} and \reffig{Gen2_Flavortriangle_nonutau} applies to the entire diffuse neutrino spectrum. With this study, flavour measurement for a given 'slice' of energy was also done to see if it has any dependence on it. Diffuse neutrinos originate from various high-energy sources in all directions. Depending on the environments of the acceleration sites (magnetic fields, accretion disks, dust, etc.), the production ratios of neutrinos at the sites may differ \todo{add proper citations here}. The most commonly assumed model, described in Chapter 1, Section 4, is the pion decay scenario.

\begin{figure}[h!]
    \centering
    \includegraphics{./figures/gen2/Gen2-10years_NONutau.png}
    \caption{Projected sensitivity of IceCube-Gen2 to measure flavor composition of Astrophysical neutrino with 10 years of its lifetime, without a dedicated $\nu_{\tau}$ identifier. The dashed (solid) outlines depict the corresponding 99\% (68\%) constraints.}
    \labfig{Gen2_Flavortriangle_nonutau}
\end{figure}

\todo{combine \reffig{Gen2_Flavortriangle} and \reffig{Gen2_Flavortriangle_nonutau} in 1 figure}
Above a critical energy, however, the flux of electron neutrinos is suppressed due to strong magnetic fields, leading to muon damping \sidecite{muon_damp} \todo{reference the muon-damped scenario, detailed in Chapter 1, Section 4}. This changes the neutrino production ratio ($\nu_e:\nu_{\mu}:\nu_{\tau}$) from 1:2:0 to 0:1:0. If such sources dominate the overall flux above a certain energy, a transition in measured neutrino flavor fluxes can be observed. \reffig{gen2_muondamped} shows IceCube-Gen2's sensitivity to detecting such a flavor transition. The assumed "critical energy" for this mechanism for the study is 2 PeV. This transition is detectable with IceCube-Gen2 due to its extended energy range, enabled by its approximately eightfold increase in volume \sidecite{Gen2_TDR}.

\begin{figure}[h!]
\centering
    \includegraphics{./figures/gen2/MuonDamped.pdf}
    \caption{The bottom section displays the proportion of $\nu_{\mu}$ at the source based on energy, with the assumption that the muon critical energy is 2 PeV. The error bars represent the 68\% confidence level limitations on the $\nu_{\mu}$ fraction below and above 1 PeV, derived from the observed flavor composition of $\nu_{\mu}$ at Earth using IceCube-Gen2 and assuming standard oscillations. In the upper sections, the dark (light) shaded regions depict the corresponding 68\% (99\%) constraints, without making any assumptions about the mixing matrix.}
    \labfig{gen2_muondamped}
\end{figure}

\appendix % From here onwards, chapters are numbered with letters, as is the appendix convention

\pagelayout{wide} % No margins
\addpart{Appendix}
\pagelayout{margin} % Restore margins


%----------------------------------------------------------------------------------------

\backmatter % Denotes the end of the main document content
\setchapterstyle{plain} % Output plain chapters from this point onwards

%----------------------------------------------------------------------------------------
%	BIBLIOGRAPHY
%----------------------------------------------------------------------------------------

% The bibliography needs to be compiled with biber using your LaTeX editor, or on the command line with 'biber main' from the template directory

\defbibnote{bibnote}{Here are the references in citation order.\par\bigskip} % Prepend this text to the bibliography
\printbibliography[heading=bibintoc, title=Bibliography, prenote=bibnote] % Add the bibliography heading to the ToC, set the title of the bibliography and output the bibliography note

%----------------------------------------------------------------------------------------
%	INDEX
%----------------------------------------------------------------------------------------

% The index needs to be compiled on the command line with 'makeindex main' from the template directory

\printindex % Output the index


\end{document}
